\section{Discussion}

A statistical significance test was conducted to examine the differences in soil characteristics between landslide and non-landslide groups. The aim of this analysis was to verify whether there were significant distinctions in soil attributes between the two groups. For this purpose, both the \textit{t}-test and Mann–Whitney \textit{U} test were employed. The use of such statistical tests follows similar methodologies adopted in prior studies. For instance, \cite{disc01} applied the Mann–Whitney \textit{U} test to compare soil properties such as clay content, bulk density, and pH between landslide-affected and unaffected areas in the Three Gorges Reservoir, while \cite{disc02} employed the Mann–Whitney \textit{U} test for non-normally distributed data and the \textit{t}-test for normally distributed data to evaluate differences in soil porosity, organic content, and texture in landslide-prone areas of the Himalayas.

The results depicted in Fig \ref{fig:Violin-plot} reveal several soil characteristics exhibiting significant differences, as indicated by p-values less than 0.05. As a prime example, variable such as \textit{T\_Cation Exchange Capacity} demonstrate statistically significant differences between the landslide and non-landslide groups. These findings are consistent with the research by \cite{disc03}, which found that landslide-prone soils had higher SOM, CEC, and Ksat in topsoil, promoting moisture retention and rapid infiltration, which favor pore pressure build-up and slope failure.

Performance indicators such as ROC, AUC, and evaluation matrices were used to validate the predictive capability of the XGBoost algorithm. These indicators follow evaluation standards similar to those employed by \cite{disc04}, who used ROC-AUC to assess landslide susceptibility mapping using XGBoost. Furthermore, the model separates the analysis between topsoil and subsoil layers, following the approach of \cite{disc05}, who investigated how different soil layers influence mass movement events.

As shown in Fig[\ref{fig:sub-top}], the comparison of model performance shows that the three experiments produce consistent AUC values. However, analyzing the F1 Scores reveals that subsoil characteristics, especially chemical composition and texture, show slightly greater predictive power than topsoil. And if we look at the combination of the two, the F1 value increases by about 3\% to reach 0.71, or about 71\%.

This concludes that although subsoil characteristics have slightly more dominant predictive power, information from topsoil also provides an essential complementary contribution to achieving the best model performance. This aligns with findings reported by \cite{disc06}, who noted that subsoil layers exhibit more stable properties such as organic content and texture over time, particularly after landslide events, thereby making them more suitable for long-term landslide analysis.

Landslide analysis using data mining approaches has proven to be a valuable tool for spatial planning and land management. However, it remains a challenge to achieve high model prediction performance using soil properties alone. As demonstrated by \cite{disc07}, emphasizing that soil-related factors (e.g., lithology) alone provide insufficient predictive power due to static nature, adding dynamic factors like annual precipitation (26\% importance) and slope significantly boosts AUC and expands high-susceptibility areas by 4.3–10.6\% under future scenarios, enabling better land management despite regional differences.

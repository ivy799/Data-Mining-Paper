\section{Discussion}

A statistical significance test was conducted to examine the differences in soil characteristics between landslide and non-landslide groups, as illustrated in the violin plots (Fig. \textit{ref}). The aim of this analysis was to verify whether there were significant distinctions in soil attributes between the two groups. For this purpose, both the \textit{t}-test and Mann–Whitney \textit{U} test were employed. The use of such statistical tests follows similar methodologies adopted in prior studies. For instance, \cite{b1} applied the Mann–Whitney \textit{U} test to compare soil properties such as clay content, bulk density, and pH between landslide-affected and unaffected areas in the Three Gorges Reservoir, while \cite{b2} employed the Mann–Whitney \textit{U} test for non-normally distributed data and the \textit{t}-test for normally distributed data to evaluate differences in soil porosity, organic content, and texture in landslide-prone areas of the Himalayas.

The results depicted in the violin plots reveal several soil characteristics exhibiting significant differences, as indicated by p-values less than 0.05. Notably, variables such as \texttt{s\_clay} and \texttt{s\_ref\_bulk density} demonstrate statistically significant differences between the landslide and non-landslide groups. These findings are consistent with the observations reported by \cite{b3}, who stated that thick clay layers with low bulk density, due to their loose structure, render the soil more susceptible to landslides.

Performance indicators such as the Receiver Operating Characteristic (ROC), Area Under the Curve (AUC), and evaluation matrices (Fig. \textit{ref}) were used to validate the predictive capability of the XGBoost algorithm. These indicators follow evaluation standards similar to those employed by \cite{b4}, who used ROC-AUC to assess landslide susceptibility mapping using XGBoost. Furthermore, the model separates the analysis between topsoil and subsoil layers, following the approach of \cite{b5}, who investigated how different soil layers influence mass movement events.

As shown in Fig. \textit{ref}, the comparison of XGBoost model performance between the two soil layers yielded an AUC of 0.6984 for topsoil and 0.7205 for subsoil. Although the difference is not statistically significant, the subsoil layer exhibited slightly superior predictive performance. This aligns with findings reported by \cite{b6}, who noted that subsoil layers exhibit more stable properties—such as organic content and texture—over time, particularly after landslide events, thereby making them more suitable for long-term landslide analysis.

An additional model evaluation was conducted by integrating both topsoil and subsoil data, as shown in Fig. \textit{ref}, resulting in a significant increase in AUC to 0.85. This suggests that combining both soil layers enhances model performance, with subsoil data contributing more strongly to the prediction. Landslide analysis using data mining approaches has proven to be a valuable tool for spatial planning and land management. However, it remains a challenge to achieve high model prediction performance using soil properties alone. As demonstrated by \cite{b7}, incorporating other environmental factors such as rainfall and slope gradients can significantly improve model performance, achieving an AUC of 0.89, even though the geographical context differs.


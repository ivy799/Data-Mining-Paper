\section{Conclusion}

The results of the study provide strong evidence that soil characteristics in landslide-prone areas differ significantly from those in stable areas. For example, statistical analysis shows that the Cation Exchange Capacity (CEC) in the topsoil layer is much higher in landslide-prone zones than in non-landslide zones. 

Feature importance analysis reinforces these findings, showing that the chemical properties of the subsoil layer, such as calcium carbonate content and electrical conductivity, play a key role in landslide prediction, highlighting the complex interaction between physical and chemical factors in triggering landslides. The results of three classification experiments confirm that although subsoil layer characteristics have slightly more dominant predictive power, information from the topsoil layer provides an essential complementary contribution to achieving optimal model performance. 

This study contributes to providing a new integrated dataset, proving significant differences in soil characteristics between the two area groups, and highlighting the distinctive contributions of each soil layer to the performance of the prediction model. These findings are expected to advance the development of soil characteristic-based early warning systems and provide valuable insights into key landslide risk indicators. Future research should consider additional environmental variables, such as rainfall intensity, slope gradient, and land cover, to build more comprehensive and spatially robust prediction models.

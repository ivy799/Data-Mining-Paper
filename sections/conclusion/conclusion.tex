\section{Conclusion}

The results provide strong evidence that soil characteristics in landslide-prone areas differ significantly from those in stable areas. For instance, statistical analysis confirmed that the Cation Exchange Capacity (CEC) in the topsoil layer is significantly higher in landslide-prone zones compared to non-landslide zones.

Feature importance analysis reinforced these findings, indicating that subsoil chemical properties such as calcium carbonate content and electrical conductivity play a pivotal role in landslide prediction. Additionally, soil textural attributes and CEC further contributed to improved model performance, underscoring the complex interplay between physical and chemical soil factors in landslide initiation. Results from the three classification experiments confirmed that although subsoil characteristics possess slightly more dominant predictive power, information from the topsoil provides an essential complementary contribution to achieving optimal model performance.

This study advances the development of soil-based early warning systems and provides valuable insights into key indicators of landslide risk. Future research should consider incorporating additional environmental variables, such as rainfall intensity, slope gradient, and land cover, to construct more comprehensive and spatially robust predictive models.
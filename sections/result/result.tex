\section{Results}
By comparing the data distribution of soil characteristics in both the landslide and no landslide groups, as illustrated in Fig[\ref{fig:Violin-plot}], it can be observed that the two groups exhibit differences in distribution, although not always significant. Taking the \textit{T\_Cation Exchange Capacity} feature as an example which represents the the soil's ability to retain cations. The no landslide group shows a data concentration between 5 and 12, with a median value tending to lie below 10. In contrast, the landslide group shows a wider and higher distribution, with a median of around 12. This statistically significant difference suggests that landslide-prone areas tend to have higher topsoil Cation Exchange Capacity.

To support this observation, a statistical analysis was conducted to calculate the \textit{p}-value using (\ref{eq:t_test}) or (\ref{eq:mann_whitney}) (the formula used depends on the data distribution, if the data distribution is normal then use t-test and u whitney-u otherwise). The \textit{T\_Cation Exchange Capacity} feature yielded a \textit{p}-value of $7.45 \times 10^{-19}$ (or 0.000000000000000000745), which is much smaller than the significance threshold of 0.05. This indicates a statistically significant difference between the two groups.

\begin{figure}[htbp]
    \centerline{\includegraphics[width=\linewidth]{fig6.png}}
    \caption{Violin Plot of Topsoil vs Subsoil.}
    \label{fig:Violin-plot}
\end{figure}

After analyzing the differences in soil characteristics between landslide and non-landslide areas, and proving that the two classes do indeed have different characteristics, the next step is to evaluate the predictive ability of the model and the contribution of both topsoil and subsoil layers to soil characteristics.

For this purpose, three separate classifications were performed. First, the model was trained using only topsoil data, second using only subsoil data, and finally a combination of both layers. The discriminative performance of these two feature sets was then evaluated using ROC-AUC metrics. The results, as shown in Fig[\ref{fig:sub-top}], indicate that the three experiments produce consistent AUC values, around 0.73 or 73\%.

\begin{figure}[htbp]
    \centerline{\includegraphics[width=\linewidth]{fig7.png}}
    \caption{ROC-AUC of Topsoil and Subsoil Layers.}
    \label{fig:sub-top}
\end{figure}

To further investigate the performance difference between the two soil layers, a feature importance analysis was conducted, as illustrated in Fig~\ref{fig:feature-importance-top-sub}. Collectively, these findings suggest that while both soil layers contribute valuable information, subsoil characteristics particularly those related to chemical composition and texture classification exhibit greater predictive power in landslide risk modeling compared to topsoil features.

\begin{figure}[htbp]
    \centerline{\includegraphics[width=\linewidth]{fig8.png}}
    \caption{Feature importance of Topsoil and Subsoil Layers.}
    \label{fig:feature-importance-top-sub}
\end{figure} 

However, analyzing the F1 Scores reveals that subsoil characteristics as shown in Table~[\ref{tab:classification_results}], especially chemical composition and texture, show slightly greater predictive power than topsoil. And if we look at the combination of the two, the F1 value increases by about 3\% to reach 0.71, or about 71\%.

\begin{table}[H]
\centering
\caption{XGBoost Classification Results for Topsoil, Subsoil, and Combined}
\label{tab:classification_results}
\begin{tabular}{lcccc}
\hline
\textbf{Model} & \textbf{Precision} & \textbf{Recall} & \textbf{Accuracy} & \textbf{F1-score (Weighted)} \\
\hline
Topsoil  & 0.7673 & 0.5982 & 0.5982  & \cellcolor{yellow!25}\textbf{0.6255} \\
Subsoil  & 0.7647 & 0.6552 & 0.6552 & \cellcolor{yellow!25}\textbf{0.6802} \\
Combined & 0.7554 & 0.6973 & 0.6973 & \cellcolor{yellow!25}\textbf{0.7153} \\
\hline
\end{tabular}
\end{table}


This concludes that although subsoil characteristics have slightly more dominant predictive power, information from topsoil also provides an essential complementary contribution to achieving the best model performance.

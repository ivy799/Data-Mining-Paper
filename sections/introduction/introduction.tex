\section{Introduction}

Landslides are among the most deadly natural disasters worldwide, particularly in tropical and mountainous regions such as South and Southeast Asia. According to data from the Center for Research on the Epidemiology of Disasters (CRED), landslides caused over 55,000 deaths globally between 2004 and 2016, with more than 75\% of these events occurring in Asia~\cite{intro01}. The primary triggers in this region include extreme rainfall, seismic activity, and complex geological conditions.
Southeast Asian countries, including Indonesia, the Philippines, and Myanmar, are especially vulnerable to landslides due to monsoonal influences and anthropogenic activities such as deforestation and slope development. As noted in~\cite{intro02}, both weathering and human-induced factors have significantly contributed to landslide occurrences in the region.
While rainfall is often considered the dominant trigger, recent studies have highlighted the critical role of soil physical properties—such as texture, permeability, moisture content, and density—in influencing slope stability and landslide susceptibility~\cite{intro03}. Nevertheless, most early warning systems still rely heavily on meteorological and topographic data, with pedological information largely overlooked.
With the advancement of data-driven techniques, machine learning has shown considerable promise in enhancing landslide risk prediction. The Extreme Gradient Boosting (XGBoost) algorithm, in particular, can effectively handle large, heterogeneous feature sets for robust classification. However, one challenge often encountered is class imbalance, where landslide instances (label 1) are significantly outnumbered by non-landslide instances (label 0). To address this, the Synthetic Minority Over-sampling Technique (SMOTE) is employed to balance the dataset prior to model training.
This study implements a pipeline-based approach incorporating SMOTE, XGBoost training, and hyperparameter optimization via Bayesian Optimization. Model evaluation is conducted using Stratified K-Fold Cross-Validation with key performance metrics such as F1-score and the Receiver Operating Characteristic–Area Under the Curve (ROC-AUC). Optimal threshold selection is also explored to improve recall as part of a broader disaster risk mitigation strategy.
Through this approach, the study aims to develop a soil-property-based landslide risk prediction model that is not only accurate but also spatially informative and practical for integration into early warning systems in landslide-prone regions.

\section{Introduction}

Landslides are one of the deadliest natural disasters in the world, especially in tropical and mountainous regions such as South Asia and Southeast Asia. According to data from the Center for Research on the Epidemiology of Disasters (CRED), landslides ranked as the fourth most frequent natural disaster in 2024\cite{intro00}. One of the most recent landslides disasters occurred in Papua New Guinea, 
where a major landslide in Enga Province resulted in one of the country’s most severe disasters in recent memory. with United Nations agencies estimating that there were around 670 fatalities. This devastating event underscores the critical need to understand the underlying causes of such disasters.

Landslides are hazardous phenomena that occur when slopes become unstable. This instability is widely attributed to the synergistic effect of an area's inherent physical conditions and the influence of external events. According to regional disaster system theory, the occurrence of a landslide is the result of this interplay between spatial static susceptibility and temporal dynamic inducibility\cite{intro01}.

These causal factors are broadly classified into two categories. The first are conditioning factors (or preparatory/non-variable factors), which represent the inherent spatial susceptibility of an area, such as its geology, soil type, topography, and lithology. The second group are the triggering factors, which are the dynamic, temporal events directly responsible for initiating the slope failure. These triggers, which include high-intensity rainfall, earthquakes, volcanic activity, and disruptive human interventions (like deforestation or improper construction), are often considered the most immediate cause of a landslide.

External factors such as heavy rainfall, earthquakes, and deforestation are often considered the main triggers. As a result, many prediction models have been developed that rely entirely on these external agents. For example, a systematic review by\cite{intro02} analyzed numerous articles on rainfall thresholds, finding that most models depend solely on meteorological data. Similarly,\cite{intro03} developed empirical thresholds using only external factors, and even advanced machine learning approaches by\cite{intro04} still focus exclusively on external rainfall metrics. 
This reliance on external triggers highlights a critical gap: while the role of triggers is well-documented, the independent predictive power of internal soil properties (such as texture, permeability, and density) is less understood. Most systems ignore these subsurface dynamics, leading to potential inaccuracies. Therefore, to better isolate and quantify the contribution of these overlooked factors, this study addresses this gap by developing a predictive model that focuses exclusively on internal soil characteristics, specifically analyzing the unique contributions of surface soil and subsoil properties to landslide risk. 

Our research focuses on two main soil layers, topsoil and subsoil, which possess distinct characteristics and hydrological functions critical to slope failure mechanisms. The topsoil, typically rich in organic matter and more porous, facilitates initial rainwater infiltration. In contrast, the subsoil is generally denser, with higher clay content and lower permeability, which impedes drainage. This structural difference is a key factor in rainfall-induced landslides. As reviewed by\cite{intro05}, water seeping through the topsoil accumulates at the interface with the less permeable subsoil during intense rainfall. This buildup elevates pore water pressure, which in turn diminishes effective stress and reduces the soil's internal shear strength, acting as a primary trigger for slope failure. The significance of this mechanism is highlighted by studies referenced in their review, which show that soil saturation from infiltration can reduce the shear modulus by up to 50\%, directly linking these layered properties to landslide initiation.

This study aims to develop an accurate landslide risk prediction model while elucidating the specific contribution of pedological features.

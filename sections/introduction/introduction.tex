\section{Introduction}
Landslides are one of the deadliest natural disasters in the world, especially in tropical and mountainous regions such as South Asia and Southeast Asia. Based on data from the Center for Research on the Epidemiology of Disasters (CRED), landslides have caused more than 55,000 deaths globally between 2004 and 2016, and more than 75\% of these events occurred in the Asian region \cite{intro01}. The main causes of landslides in this region are extreme rainfall, seismic activity and complex geology.
The Southeast Asian region, including Indonesia, the Philippines and Myanmar, is particularly vulnerable to landslides due to the influence of the monsoon climate as well as anthropogenic activities such as deforestation and development on steep slopes. According to \cite{intro02},  that weathering and human activities at the landslide site in the research location contributed significantly to the occurrence of landslides.
While rainfall is often considered the main trigger, recent research has shown that the physical characteristics of the soil-such as texture, permeability, moisture content and density-contribute significantly to slope stability and landslide potential \cite{intro03}. However, most early warning systems are still dominated by meteorological and topographic-based approaches, while pedological information is often neglected \cite{intro04}.
Along with the development of data-driven approaches, machine learning offers great potential in improving the accuracy of landslide risk prediction. XGBoost Algorithms can utilize hundreds of soil and environmental features in robust classification models. However, challenges such as class imbalance often arise, as the amount of landslide data (label 1) is generally much less than non-landslide data (label 0). To overcome this, the SMOTE (Synthetic Minority Over-sampling Technique) is used, which is effective in balancing the data distribution before model training.
This study adopts a pipeline-based approach with SMOTE integration, model training with XGBoost and hyperparameter tuning using Bayesian Optimization. Model evaluation is conducted through Stratified K-Fold Cross Validation with key metrics such as F1-score and ROC-AUC, and exploration of optimal thresholds to improve recall as part of disaster risk mitigation strategies.
Through this approach, the research aims to build a soil property-based landslide risk prediction model that is not only accurate, but also spatially informative and practical to be implemented in early warning systems in disaster-prone areas.
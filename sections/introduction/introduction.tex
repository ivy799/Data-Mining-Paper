\section{Introduction}

Landslides are one of the deadliest natural disasters in the world, especially in tropical and mountainous regions such as South Asia and Southeast Asia. According to data from the Centre for Research on the Epidemiology of Disasters (CRED), landslides caused more than 55,000 deaths globally between 2004 and 2016, with more than 75\% of these events occurring in Asia~\cite{intro01}. 
The main triggers in these regions include extreme rainfall, seismic activity, and vulnerable geological conditions. 

Although those factor is often considered the dominant trigger as noted in~\cite{intro02}, recent studies highlight the critical role of soil physical properties such as texture, permeability, and density in influencing slope stability~\cite{intro03}. 
However, most existing early warning systems still rely heavily on meteorological and topographical data, while pedological information is often overlooked.

This study focuses on two main soil layers, namely topsoil (upper layer) and subsoil (lower layer), both of which have different characteristics and hydrological functions in slope failure mechanisms. Topsoil is rich in organic matter and more porous, functioning as the first layer to receive and infiltrate rainwater. In contrast, subsoil tends to be denser, has a higher clay content, and lower permeability. 
During periods of intense rainfall, water that seeps through the topsoil can accumulate above the subsoil layer, causing a significant increase in pore water pressure. This increase in pressure reduces the internal shear strength of the soil, which is a key factor in triggering landslides.

To analyze the relationship between soil characteristics in both topsoil and subsoil layers in landslide and non-landslide areas, this study applied a two stage approach. The first stage was an inferential statistical analysis to test the significance of differences in soil properties between landslide and non-landslide locations. 
For this purpose, an independent t-test was applied to normally distributed data and a non-parametric Mann-Whitney U test was applied to data that did not meet the assumption of normality. These two tests served to quantitatively validate whether a soil feature had a significant difference between the two groups, so that it could be considered a relevant distinguishing factor. 

The second stage is predictive modeling using machine learning algorithms. 
Based on statistically significant features, an Extreme Gradient Boosting (XGBoost) model was developed to predict the probability of landslide risk. This algorithm was chosen for its proven superior performance, ability to model complex feature interactions, and resistance to overfitting.

However, one of the main challenges in this modeling is the class imbalance in the dataset, where the amount of non-landslide data is much greater than the landslide data. To overcome this problem, an oversampling technique using the Synthetic Minority Over-sampling Technique (SMOTE) was applied. 
Next, the XGBoost model was trained using a balanced dataset, with hyperparameter optimization performed through Bayesian Optimization to achieve maximum performance. Model evaluation was carried out using a Stratified K-Fold Cross-Validation scheme to ensure robustness, with key performance metrics including F1-score and Receiver Operating Characteristic Area Under the Curve (ROC-AUC).
Through this approach, the study aims to develop a landslide risk prediction model based on soil properties that is not only accurate but also provides an overview of the contribution of pedological features. The results of this study are expected to form the basis for future research to integrate soil data with other key triggers, such as meteorological and topographical data, in order to improve early warning systems in the future.
